\documentclass[margin,line]{res}


\oddsidemargin -.5in
\evensidemargin -.5in
\textwidth=6.0in
\itemsep=0in
\parsep=0in
% if using pdflatex:
\setlength{\pdfpagewidth}{\paperwidth}
\setlength{\pdfpageheight}{\paperheight} 

\newenvironment{list1}{
  \begin{list}{\ding{113}}{%
      \setlength{\itemsep}{0in}
      \setlength{\parsep}{0in} \setlength{\parskip}{0in}
      \setlength{\topsep}{0in} \setlength{\partopsep}{0in} 
      \setlength{\leftmargin}{0.17in}}}{\end{list}}
\newenvironment{list2}{
  \begin{list}{$\bullet$}{%
      \setlength{\itemsep}{0in}
      \setlength{\parsep}{0in} \setlength{\parskip}{0in}
      \setlength{\topsep}{0in} \setlength{\partopsep}{0in} 
      \setlength{\leftmargin}{0.2in}}}{\end{list}}
\usepackage{url}

\begin{document}

\name{Frank F. Xu \vspace*{.1in}}

\begin{resume}
\section{\sc Contact Information}
\vspace{.05in}
\begin{tabular}{@{}p{3in}p{4in}}
SEIEE Bldg. 3-341             & {Phone:}  (+86) 188-1827-5852 \\            
Department of Computer Science \& Engineering   & { E-mail:}  \url{frankxu@sjtu.edu.cn} \\         
Shanghai Jiao Tong University & {URL:} \url{frankxfz.me}\\       
800 Dongchuan Road   & {Skype:} \url{fzxu2004}\\  
Shanghai 200240, China & {Github:} \url{github.com/frankxu2004}
\end{tabular}


\section{\sc Research Interests}
Natural Language Processing, Text Mining, Knowledge Discovery, Computational Social Science, Urban Computing, Machine Learning, Artificial Intelligence

\section{\sc Education}
{\bf Shanghai Jiao Tong University}, Shanghai, China\\
%{\em Department of Statistics} 
\vspace*{-.1in}
\begin{list1}
\item[] M.S. Candidate, Computer Science, September 2016 (expected
  graduation date: May 2019)
\begin{list2}
\vspace*{.05in}
\item Research Field: Natural language processing, text mining, knowledge discovery, spatial-temporal data mining
%\item Dissertation Topic:  ``Hierarchical Models for Multiple Ratings
%  in Performance-Based\\ \hspace*{1.23in} Student Assessments.'' 
\item Advisor:  Prof. Kenny Q. Zhu
\end{list2}
\vspace*{.05in}
\item[] B.Eng., Computer Science (IEEE Honored Class), June 2016
\begin{list2}
	\vspace*{.05in}
	\item Thesis Topic: ``Traffic Prediction for Urban Planning''
	\item Advisor: Prof. Kenny Q. Zhu
\end{list2}
\end{list1}

%{\bf Duke University}, Durham, North Carolina USA\\
%%{\em Department of Mathematics and Statistics} 
%\vspace*{-.1in}
%\begin{list1}
%\item[] M.S., Botany (Ecology),  May, 1998
%\end{list1}
%
%{\bf Carleton College}, Northfield, Minnesota USA\\
%%{\em Department of Mathematics and Statistics} 
%\vspace*{-.1in}
%\begin{list1}
%\item[] B.A., Biology,  May, 1993
%\end{list1}


\section{\sc Honors and Awards} 
Best Thesis Award in Bachelor's Thesis Defense (Top 1\%), Shanghai Jiao Tong University, 2016

\vspace*{-2.5mm}
Second Place in EMC Smart Campus Open Data Contest, Shanghai Jiao Tong University, 2015

%\vspace*{-2.5mm}
%Carleton College: graduated Magna Cum Laude, Honors in Biology, Phi Beta Kappa, 1993

\section{\sc Academic Experience}
{\bf Shanghai Jiao Tong University}, Shanghai, China

\vspace{-.3cm}
{\em Graduate Student and Research Assistant at ADAPT Lab} \hfill {\bf September, 2016 - present}\\
\url{http://adapt.seiee.sjtu.edu.cn/}\\
Working at ADAPT Lab, supervised by Prof. Kenny Q. Zhu, collaborate with Bill Y. Lin, work includes current Master research, coursework and
research or developing projects.

%\vspace{-.1cm}
{\em Teaching Assistant} \hfill {\bf September, 2016 - January, 2017}\\
Various duties on undergraduate level course for Computer Science (IEEE Honored Class). Shared responsibility for tutorials, exams, homework assignments, and grades. This is an English-only class.
\begin{list2}
\item SE305 Database System Technology, Fall 2016.
\end{list2}

{\em Research Intern at SJTU IIOT Lab} \hfill {\bf January, 2015 - July, 2015}\\
\url{http://iiot.sjtu.edu.cn}\\
Worked at SJTU IIOT Lab, led by Prof. Xinbing Wang. Joining a team on AceMap: Academic Search and Paper Recommendation System and mainly contributed to academic cralwer, paper topic model, search algorithms and infrastructure.

{\em Research Intern at OMNILAB} \hfill {\bf June, 2014 - December, 2014}\\
\url{http://omnilab.sjtu.edu.cn}\\
Worked at OMNILAB, led by Prof. Yaohui Jin, located at Netword and Information Center in SJTU. Contributed to open data platform and data mining projects from web crawlers to data analysis and visualizations.
\section{\sc Research Projects}
{\bf Bilingual Word Representations with Cross-cultural Socio-linguistic Features}

\vspace{-.3cm}
Capturing cross-cultural differences is an important challenge in bilingual text understanding and machine translation. This paper presents a novel framework for obtaining bilingual word representations from social media by leveraging socio-linguistic features. Such representations can act as a building block for cross-lingual studies in computational social science. We evaluated our framework on two such tasks: detection of cross-cultural differences in named entities and bilingual lexicon extraction for Internet slangs. Experimental results show that the proposed word representations outperform a number of baseline methods by substantial margins. We have submitted this work to IJCAI 2017.

{\bf Cross-region Traffic Prediction for China on OpenStreetMap}

\vspace{-.3cm}
This is an interdisciplinary study of computer science, transportation and urban planning. We built a system to learn a prediction model from graphical traffic condition data, and thus everyone can predict the traffic conditions with nearly 90\% accuracy in Shanghai, China, even if no historical traffic data is available for that area. This novel system is useful in urban planning, transportation dispatching as well as personal travel planning. We have published this work in 9th ACM SIGSPATIAL International Workshop on Computational Transportation Science. We are still improving it with many innovative ideas.

{\bf AceMap: Academic Search and Paper Recommendation System}

\vspace{-.3cm}
\url{http://acemap.sjtu.edu.cn}\\
A team work led by Prof. Xinbing Wang. Our goal is to build an academic search engine which can return paper search results based on topic similarity with user's query, analyze the latent topic distribution and topic development over time, visualize the ``topic tree'' starting from a particular paper and more. It is like an academic social network with features like search, academic maps, paper network, etc. We are currently implementing machine learning like LDA and HLDA, NLP algorithms and network analysis algorithms for processing and analyzing paper database which is crawled from various sources, and expose the search feature with the help of the robust Apache Solr with many customizations. I have been working on Web development, data visualization with D3.js, some Apache Solr configuration, and paper citation analysis of conference datasets and clustering algorithms for Academic Map feature.


\section{\sc Publications}
\textbf{Bilingual Word Representations with Cross-cultural Socio-linguistic Features.}\\
\textbf{Frank F. Xu\textsuperscript{*}}, Bill Y. Lin\textsuperscript{*}, Hanyuan Shi, Kenny Q. Zhu and Seung-won Hwang.\\
\textit{Under review of Twenty-sixth International Joint Conference on Artificial Intelligence}, 2017 (\textit{* indicates equal contribution})

\textbf{Cross-region Traffic Prediction for China on OpenStreetMap.}\\
\textbf{Frank F. Xu}, Bill Y. Lin, Qi Lu, Yifei Huang and Kenny Q. Zhu.\\
\textit{In the Proceedings of the 9th ACM SIGSPATIAL International Workshop on Computational Transportation Science}, 2016

%
%\section{\sc Conference Presentations}
%
\section{\sc Industry Experience}
{\bf Baixing.com}, Shanghai, China

\vspace{-.3cm}
{\em Software Developer Intern} \hfill {\bf August, 2015 - May, 2016}\\
Developed open APIs, crawlers, backend tools, and some modules to prevent spam and inappropriate posts, like posts filtering and classification, image identification. Mainly used Python 3, Tornado, MongoDB and Redis.

\section{\sc Programming Projects}
{\bf EMC Smart Campus Open Data Contest}

\vspace{-.3cm}
\url{https://github.com/frankxu2004/emc-contest-data-visualization}\\
Our team, who mined and analyzed the data of wireless network in the SJTU campus, came the second in this contest. We implemented features like "Are You Really Studying", finding friends based on interests, flow and crowd mapping and statistics. Cluster analysis, trajectory analysis and many other types of machine learning and data mining algorithms like SVM, random forest are used. I was mainly responsible for data processing, statistics and data visualization. The web platform is built using Python Flask and SQLite as backend Restful API and AngluarJS and Highcharts as frontend Web App.

{\bf OMNILab Open Data Platform}

\vspace{-.3cm}
\url{http://data.sjtu.edu.cn}\\
\url{https://github.com/frankxu2004/AQICrawler}\\
An open data platform based on the open-source CKAN data platform, is built by our lab OMNILab in Shanghai Jiaotong University led by Professor Yaohui Jin of Network and Information Center. We made many modifications to the original CKAN project, from backend data processing API and talking to PostgreSQL, to front-end data visualization, geo-infomation and search. I myself wrote a crawler named AQICrawler for air quality data from various sources like \textit{pm25.in}, \textit{aqicn.org} and government site and upload the data to our open data platform through APIs.

{\bf DivineMove}

\vspace{-.3cm}
\url{https://github.com/frankxu2004/DivineMove}\\
An AI for Go Game team project, written in C with an open protocol for Computer Go,GTP, and using effective, proven algorithms like Monte-Carlo simulation and UCT tree search. We made several optimizations and modifications to the original algorithms presented in papers such as cutting branches in UCT tree, reusing tree structure, "filling water" method for fast observation of game situiation and better Monte-Carlo simulations.


{\bf WeixinCrawler}

\vspace{-.3cm}
\url{https://github.com/frankxu2004/WeixinCrawler}\\
A Weixin public account crawler written in Python. It fetches articles of interested public accounts from \textit{weixin.sogou.com} and stores them into MySQL database. The API for generating RSS feed for the articles and a simple page for adding new accounts are written in PHP.

{\bf LyricsSearch}

\vspace{-.3cm}
\url{https://github.com/frankxu2004/lyrics-search}\\
An music search engine by lyrics with PyLucene and web.py with some level of fuzzy query. The data is crawled with a Python crawler from Baidu MP3 for songs and lyrics and then processed and indexed by Lucene for search.

{\bf Feedy}

\vspace{-.3cm}
\url{https://github.com/SJTUCat/Feedy}\\
A Chrome extension developed by me and Easton Wang. It is a feed notifier and reader for custom RSS or Atom feeds, using Google Feed API.

\section{\sc Computer Skills} 
\begin{list2}
\item Machine Learning and Deep Learning Packages: Scipy, scikit-learn, TensorFlow, Theano, Keras
\item Natural Language Processing Tools: Stanford CoreNLP, NLTK
\item Distributed Platforms: Hadoop, Spark
\item Databases: MySQL, PostgreSQL, neo4j
\item Languages: Python, Java, C/C++, PHP, Javascript/Node.js, HTML5, some use of Linux shell scripts,
  CUDA/MPI parallel processing library
\item Applications: GNUPlot, \LaTeX, common spreadsheet and presentation software
\item Operating Systems:  Unix/Linux, Windows\\ 
\end{list2}

\end{resume}
\end{document}




